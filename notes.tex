\documentclass{article}
\usepackage{style}
\usepackage{quiver}

\RequirePackage[a4paper]{geometry}
\geometry{tmargin=3cm, bmargin=3cm, lmargin=2cm, rmargin=3cm}

% links
\RequirePackage{hyperref}
\hypersetup{
  colorlinks,
  citecolor=black,
  filecolor=black,
  linkcolor=black,
  urlcolor=black
}

% headers
\RequirePackage{fancyhdr}
\pagestyle{fancy}
\fancyhead[L]{\leftmark}
\fancyhead[R]{\thepage}
\fancyfoot[C]{}
\renewcommand{\headrulewidth}{0pt}

\author{Franz Miltz}
\title{Sheaf theory}

\begin{document}
\maketitle
\tableofcontents
\pagebreak

\printbibliography

\section{Sections}\label{sec:sections}

\begin{definition}
  \missingdef
\end{definition}

\missingexample

\section{Categories and functors}\label{sec:categories_and_functors}

\begin{definition}[Category]\label{def:category}
  \missingdef
\end{definition}

\begin{definition}[Functor]\label{def:functor}
  \missingdef
\end{definition}

\section{Sheaves}\label{sec:sheaves}

\begin{definition}[Sheaf]
  \missingdef
\end{definition}

\begin{definition}[Constant sheaf]
  \missingdef
\end{definition}

\begin{definition}[Locally constant sheaf]
  \missingdef
\end{definition}

\end{document}
