\documentclass{article}
\usepackage{style}

\author{Franz Miltz}
\title{Sheaf theory}

\begin{document}
\maketitle
\tableofcontents
\pagebreak

\printbibliography

\section{Categories and functors}\label{sec:categories_and_functors}

Categories generalise the notion of algebraic objects and their homomorphisms.
Rather than focusing on the internal structure of the objects, we study the
morphisms between them.

\begin{definition}\label{def:category}
  A \emph{category} $\cat C$ consists of
  \begin{enumerate}
    \item a class of objects $\Ob\cat{C}$;
    \item for all $X,Y\in\Ob\cat{C}$, a class of arrows $\cat{C}\rr{X,Y}$;
    \item for all $X\in\Ob\cat{C}$, an arrow $1_X\in\cat{C}\rr{X,X}$;
    \item for all $f\in\cat{C}\rr{X,Y}$ and $g\in\cat{C}\rr{Y,Z}$, an arrow
      $g\circ f\in\cat{C}\rr{X,Z}$
  \end{enumerate}
  such that
  \begin{enumerate}
    \item for all $f\in\cat{C}\rr{X,Y}$, $f\circ 1_X = f = 1_Y\circ f$;
    \item for all $f\in\cat{C}\rr{W,X}$, $g\in\cat{C}\rr{X,Y}$, $h\in\cat{C}\rr{Y,Z}$,
      $h\circ(g\circ f)=(h\circ g)\circ f$.
  \end{enumerate}
\end{definition}

\begin{notation}
  We improve notation as follows:
  \begin{itemize}
    \item $X\in\cat{C}$ means $X\in\Ob\cat{C}$;
    \item $f:X\to Y$ in $\cat{C}$ means $X,Y\in\Ob\cat{C}$ and $f\in\cat{C}\rr{X,Y}$;
    \item for composable arrows $f$ and $g$ in $\cat{C}$, we may write $gf$ to mean $g\circ f$.
  \end{itemize}
\end{notation}

\begin{example}
  There are some common algebraic structures that form categories with their respective
  homomorphisms:
  \begin{itemize}
    \item $\text{Set}$: objects are sets, arrows are functions;
    \item $\text{Hilb}$: objects are Hilbert spaces, arrows are bounded linear operators;
    \item $\text{Grp}$: objects are groups, arrows are homomorphisms
  \end{itemize}
\end{example}

\begin{example}
  However, we also have categories where the arrows are not functions of any kind:
  \begin{itemize}
    \item $\text{Rel}$: objects are sets, arrows are relations $R\subseteq X\times Y$,
      composition is
      \begin{align*}
        S\circ R=\cc{\rr{x,z} : \exists y.\:\rr{x,y}\in R,\rr{y,z}\in S}
      \end{align*}
    \item For any poset $(P,\leq)$, we have a category with objects $x\in P$,
      a unique arrow $x\to y$ iff $x\leq y$, and composition of $x\to y$ and
      $y\to z$ yields the unique arrow $x\to z$ which exists by transitivity
      of $\leq$.
    \item Using the above, for any topological space $X$, the poset of open sets
      $(\text{Opens}X, \subseteq)$ may be regarded as a category.
  \end{itemize}
\end{example}

\begin{definition}\label{def:functor}
  Let $\cat{C},\cat{D}$ be categories. A \emph{functor} $F:\cat{C}\to\cat{D}$
  consists of
  \begin{enumerate}
    \item for each $X\in\cat{C}$, an object $FX\in\cat{D}$;
    \item for each $f:X\to Y$ in $\cat{C}$, an arrow $Ff:FX\to FY$
  \end{enumerate}
  such that
  \begin{enumerate}
    \item for all $X\in\cat{C}$, $F1_X=1_{FX}$;
    \item for all composable arrows $f,g$ in $\cat{C}$, $F\rr{g\circ f}=Fg\circ Ff$.
  \end{enumerate}
\end{definition}

\begin{example}
  We now have another category, $\Cat$, with objects all small categories, and
  arrows are functors. Moreover, there is an obvious inclusion functor
  $J:\Cat\to\CAT$.
\end{example}

\missingexample
\note{Add more concrete examples}

\section{Sections}\label{sec:sections}

\begin{definition}
  Let $X,Y$ be topological spaces and $p:Y\to X$ be a continuous map. A continuous section
  is a continuous map $s:X\to Y$ such that $p \circ s = 1_X$.

  More generally, let $p:Y\to X$ be an arrow in a category $\cat{C}$. Then a section
  is an arrow $s:X\to Y$ such that $p\circ s=1_X$ in $\cat{C}$.
\end{definition}

\begin{remark}
The existence of a section implies that the map $p$ is surjective. This is obvious from the definition.
The converse depends on the axiom of choice. If $p$ is surjective, then there exists a section of $p$ if we accept the axiom.
\end{remark}

\begin{definition}
  Let $X,Y$ be a topological space, let $U\subseteq X$ be open, and let $p:Y\to X$ be
  a continuous map. Then the \emph{sheaf of sections of $p$ at $U$} is the set
  $\Gamma\rr{U,p}$ of all continuous maps $s:U\to X$ such that, for all $x\in U$,
  $\rr{p\circ s}\rr{x}=x$.
\end{definition}

\begin{example}
Let let $p : \mathbb{R} \to \mathbb{R}$ be the continuous map $x \mapsto x^2$. We have that
\begin{align*}
	\Gamma\rr{U,p} = \{\text{set of square root functions on } U\}.
\end{align*}
For example, if $U = ]1, 2[$ then
\begin{align*}
	\Gamma\rr{U,p} = \{+\sqrt{\phantom{a}}, -\sqrt{\phantom{a}}\}.
\end{align*}
In particular, a square root function can be described as a continuous map $\sigma  : U \to \mathbb{R}$ such that for every  $u \in U$ we have $\sigma(u)^2 = u$.
\end{example}

\begin{example}
\label{ex:complexroot}
Consider the continuous map $p : \mathbb{C}^\times \to \mathbb{C}^\times$ sending $z \mapsto z^2$. Note that this map is not continuous if considered over all of $\mathbb{C}$. If we "confine ourselves to a quadrant on the circle" (i.e. open sets $U$ contained in one quadrant) we have that
\begin{align*}
	\# \Gamma\rr{U, p} = 2.
\end{align*}
This makes sense, as the inverse function $z^{1/2}$ is a multivalued function with exactly $2$ values. What we mean by the "confinement to the quadrant" is that if we happened to have a union of open sets that somehow "cross" the removed point $0$, then the sheaf of sections will have no sections at all. This makes sense insofar as such a section would not be continuous near $0$. In particular, the sheaf of scetion seems to know how the open sets are distributed.
\end{example}
\note{This example is a bit poorly explained, but I think is good to consider for intuition, let's discuss how to fix this? Come back to this example and how it becomes problematic when gluing - disagreement on overlap}

\missingexample
\note{add more complicated example}


For two topological spaces $X$ and $Y$ and a continuous map $p : X \to Y$, sheaves of sections give us information about how homeomorphic the map $p$ is.

\begin{proposition}
Let $p : X \to Y$ be a continuous map. If $p$ is a homeomorphism then $\Gamma\rr{U,p}$ consists of the continuous map $g : Y \to X$ such that $g \circ f = 1_X$ and $f \circ g = 1_Y$. 
\end{proposition}

\begin{remark}
We will note this formally later, however something to think about is that if a map $p : X \to Y$ is locally homeomorphic then we can recover $p$ from its sheaf of sections. In particular, we will see that this relates to the concept of "gluing" things together.
\end{remark}

\section{Sheaves}\label{sec:sheaves}

\begin{definition}
  Let $X$ be a topological space. A \emph{presheaf on $X$} is a functor
  $F:\Open\rr{X}^{\text{op}}\to\Set$.
\end{definition}

\begin{notation}
  Following the sheaf of sections, we adapt the following:
  \begin{enumerate}
    \item for opens $U\subseteq X$, we write $\Gamma\rr{U,F}=F$;
    \item for inclusions $j:U\to V$ in $\Open\rr{X}$, $\res_{U,V}=Fj$.
  \end{enumerate}
\end{notation}

\begin{example}[Restriction Map]
Let $X, Y$ be topological spaces, and let $U \subseteq V \subseteq X$ be open. Suppose $p : Y \to X$ is continuous. Then we can map from $\Gamma\rr{V,p}$ to $\Gamma\rr{U,p}$ using the restriction map from $U$ to $V$:
\begin{align*}
	\res_{U,V} : \Gamma\rr{V,p} &\to \Gamma\rr{U,p}\\
	\sigma &\mapsto \sigma|_U.
\end{align*}
\end{example}

We note that this map does not have to be injective or surjective.

\begin{definition}
  Let $X$ be a topological space. A \emph{sheaf on $X$} is a presheaf $F$ on $X$
  such that
  \begin{enumerate}
    \item for every open cover $\cc{U_i}_{i\in I}$ of an open set $U\subseteq X$ and
      $s,t\in\Gamma\rr{U,F}$,
      \begin{align*}
        \forall i,j\in I.\: \res_{U_i,U}(s) = \res_{U_j,U}(t)
      \end{align*}
      implies $s=t$;
    \item for every open cover $\cc{U_i}_{i\in I}$ of an open set $U\subseteq X$
      and every family $\cc{s_i}_{i\in I}$ such that $s_i\in FU_i$, for all $i$,
      if, for all $i,j\in I$, $\restriction{s_i}{U_i\cap U_j} = \restriction{s_j}{U_i\cap U_j}$,
      then there exists an $s\in FU$ such that, for all $i\in I$,
      $\restriction{s}{U_i}=s_i$.
  \end{enumerate}
\end{definition}

\begin{definition}
  Let $F,G$ be sheaves on $X$. A \emph{homomorphism $\eta:F\to G$} is
  a natural transformation $F\Rightarrow G$.
\end{definition}

\begin{definition}
  Let $X$ be a topological space. The \emph{category of sheaves $\textbf{Sheaf}(X)$}
  has as objects all sheaves on $X$ and as arrows all natural transformations between
  them.
\end{definition}

\section{Locally constant sheaves}\label{sec:locally_constant_sheaves}

We say a sheaf is constant if it is constant on every connected component. We choose this
definition because an assignment $U\mapsto S$ cannot satisfy the the sheaf axioms.

\begin{definition}
  Let $S$ be a set and $X$ a topological sapce. Then
  the constant sheaf $F^S$ is the sheaf of sections
  of the projection $\pi:S\times X\to X$.
\end{definition}


\begin{definition}[Constant sheaf]
  A sheaf $F$ on $X$ is constant if, for some set $S$ and all opens $U\subseteq X$,
  \note{verify}
  \begin{equation}
    \Gamma(U,F) = \prod_{\pi_0U}S
  \end{equation}
  where $\pi_0U$ is the set of connected components of $U$.
\end{definition}

\begin{definition}
  A \emph{locally constant sheaf $F$ on $X$} is a sheaf such that,
  for all $x\in X$, there exists an open neighbourhood $U$ of $x$,
  a set $S_x$, and an isomorphism
  \begin{align*}
    \text{res}_{U,X} \cong \restriction{F^{S_x}}{U}.
  \end{align*}
\end{definition}

\begin{lemma}
  Let $B$ be a basis of $X$ and let $F,G$ be sheaves on $X$. Suppose, for all $U\in B$,
  we have a map $\eta_U:FU\to GU$ such that, for all $V\subseteq U$,
  \begin{equation}
    % https://q.uiver.app/?q=WzAsNCxbMCwwLCJGVSJdLFswLDEsIkdVIl0sWzIsMCwiRlYiXSxbMiwxLCJHViJdLFswLDIsIkZpIl0sWzAsMSwiXFxldGFfVSIsMl0sWzIsMywiXFxldGFfViJdLFsxLDMsIkdpIiwyXV0=
    \begin{tikzcd}
      FU && FV \\
      GU && GV
      \arrow["Fi", from=1-1, to=1-3]
      \arrow["{\eta_U}"', from=1-1, to=2-1]
      \arrow["{\eta_V}", from=1-3, to=2-3]
      \arrow["Gi"', from=2-1, to=2-3]
    \end{tikzcd}
  \end{equation}
  Then there exists a unique sheaf homomorphism $\tilde\eta : F\to G$.
  \begin{proof}
    \missingproof
  \end{proof}
\end{lemma}

\begin{theorem}
  Let $F$ be a locally constant sheaf. Then there exists a covering space
  $p:E\to X$ such that $\restriction{F}{E} \cong F$.
  \begin{proof}
    \missingproof
  \end{proof}
\end{theorem}

\begin{example}
  Take $p:S^1 \to S^1$ as $z\mapsto z^k$ for $k\geq 2$.
  Then $\Gamma\rr{-,p}$ is a locally constant sheaf.
  \begin{proof}
    Let $U \in S^1$ be open. It suffices to prove this statement for $U$ of the form $U = ]x - \epsilon, x + \epsilon[$ for some $x \in S^1$ and $\epsilon \ll 1$. From complex analysis we know that there are $k$ points satisfying $z = z^k$, hence $p^{-1}(x) = \{x_1, ..., x_k\}$. In particular, we have that $p^{-1}(U) = \amalg_{i=1}^k ]x_i - \frac{\epsilon}{k}, x_i + \frac{\epsilon}{k}[$. Since each interval is disjoint, we have that $p^{-1}(U) \cong U \times \mathbb{Z}/k\mathbb{Z}$. It therefore follows that locally we have a constant sheaf.
\missingproof
\note{This is a sketch, I think it would be good to clarify some points, i.e. the isomorphism to the product, and the disjoint union with the epsilon stuff}
  \end{proof}
\end{example}

\begin{nonexample}
 Take $p: \mathbb{R} \to \mathbb{R}$ as $x \mapsto (x-1)(x+1)$.
  Then $\Gamma\rr{-,p}$ is not a locally constant sheaf.
  \begin{proof}
    \missingproof
  \end{proof}
  \note{Check the map types not sure if correct}
\end{nonexample}

\begin{lemma}
  If $F$ is a constant sheaf on $X$ and $V\subseteq U\subseteq X$ are opens then
  \begin{align*}
    \Gamma\rr{U, F} = \Gamma\rr{V,F}.
  \end{align*}
\end{lemma}

\begin{example}
  Let $X=\R$ and let $\sim$ be the equivalence relation on $\R+\R$ such that
  $x\sim y$ iff $x=y\neq 0$ and let $Y=\rr{\R+\R}/\sim$. Let $p:Y\to X$ be
  given by the identity. Then the sheaf of sections $\Gamma\rr{-,p}$
  is not locally constant.
\end{example}

\section{Covering space}\label{sec:covering_space}

\begin{definition}
  Let $i:Y\to X$ an open map and $F$ a sheaf on $X$. Then
  the \emph{pullback along $i$} is the sheaf $i^*F$ on $Y$
  given by, for all $U\subseteq Y$,
  \begin{align*}
    i^*F(U) = F(i(U)).
  \end{align*}
\end{definition}



\begin{definition}
  A \emph{covering space} for $X$ consists of
  \begin{enumerate}
    \item a topological space $\tilde X$;
    \item a map $p:\tilde X\to X$
  \end{enumerate}
  such that, for all $x\in X$, there are sets $U,S$ with $x\in U$, $\inv p\rr{U}=S\times U$,
  and, for all $s\in S$, $\restriction{p}{\cc{s}\times U}:\cc{S}\times U\to U$ is a
  homeomorphism.
\end{definition}

\begin{notation}
  We denote a covering $(\tilde X,p)$ of $X$ by $p:\tilde X\to X$.
\end{notation}

\begin{example}
  The map $p:S^1\to S^1$ given by $z\mapsto x^k$, for some $k$, is a covering.
  \begin{proof}
    \missingproof
  \end{proof}
\end{example}

\begin{example}
  Let $X$ be two copies of $S^1$ that intersect in a single point. Weirdly connected
  graph?
\end{example}

\begin{example}
  The map $p:\R^2\to\R$ given by $\rr{x,y}\mapsto x$ is not a covering.
  \begin{proof}
    \missingproof
  \end{proof}
\end{example}

\begin{example}
  The canonical map $p:\R^2\to\R^2/\Z$ is a covering.
  \begin{proof}
    \missingproof
  \end{proof}
\end{example}

\begin{proposition}
  Let $p:\tilde X\to X$ and $p':\tilde{\tilde X}\to\tilde X$ be coverings.
  Then $pp':\tilde{\tilde{X}}\to X$ is a covering.
  \begin{proof}
    \missingproof
  \end{proof}
\end{proposition}

\begin{definition}
  Let $p_1:\tilde X_1\to X$ and $p_2:\tilde X_2\to X$ be coverings. A covering homomorphism
  is a map $f:\tilde X_1\to\tilde X_2$ such that $p_1 = p_2\circ f$.
\end{definition}

\begin{definition}
  A path-connected covering of $X$ is a covering $(\tilde X,p)$ where $\tilde X$ is
  path-connected.
\end{definition}

\begin{theorem}
  If $X$ is locally path connected then there exists a cover
  $(\tilde X,p)$ such that, for all other covers $(Y,q)$ and choice of $x\in X$, $y\in Y$
  with $p(y)=x$, there exists a unique homorphism of covers $f:(\tilde X,p)\to(Y,q)$
  with \missingdef
\end{theorem}

\begin{theorem}
  Let $p:\tilde X\to X$ be a covering. If $\tilde X$ is simply connected then
  $\tilde X$ is the universal cover.
  \begin{proof}
    \missingproof
  \end{proof}
\end{theorem}

\begin{definition}
  Let $F$ be a sheaf on $X$ and let $x\in X$. The stalk $F_x$ of $F$ at $X$ is the
  colimit of the sheaf applied to the category of open sets containing $x$.
\end{definition}

\begin{example}
  \begin{itemize}
  \item $(C^\infty)_0$.
  \end{itemize}
\end{example}

\begin{lemma}
  Let $F$ be a locally constant sheaf on $X$. Then, for all $x\in X$, there exists
  an open $U\subseteq X$ with $x\in U$ such that $FU=F_x$.
  \begin{proof}
    \missingproof
  \end{proof}
\end{lemma}

\section{Geometry}

\begin{definition}
  An \emph{$n$-dimensional $C^k$-manifold} consists of
  \begin{enumerate}
    \item a topological space $X$;
    \item a sheaf of ring $\mathcal O_X:\Open(X)^{\text{op}}\to\textbf{Rings}$
  \end{enumerate}
  such that there exists an open cover $\cc{U_\alpha}_{\alpha\in\Lambda}$
  together with isomorphisms
  \begin{align*}
    (U_\alpha,\restriction{U_\alpha}{\mathcal O_X})\cong (\R^n,C^k).
  \end{align*}
\end{definition}

\begin{example}
  \begin{itemize}
    \item $(\text{Spec} R,\mathcal O_{\text{Spec} R})$;
    \item $(\C^n, \text{holos})$;
    \item $(\R^n, C^\infty)$.
  \end{itemize}
\end{example}

\printbibliography

\end{document}
